\Chapter{Bevezetés}

A 21. században az eddigieknél, sokkal nagyobb fejlődést mutató, egyre inkább digitalizálódó világban a különböző akár mindennapi adminisztrációk egyre nagyobb részét végezzük digitális felületeken.
A papír alapú formanyomtatványokat már több helyen leváltották, a dinamikusan fejlődő és könnyedén változtatható WEB-es felületek. A dossziékat és irattartókat pedig adatbázisokra cserélték.
Ezáltal új lehetőségek nyíltak a munkáltatók számára, hogy a bonyolult, időigényes, személyes ügyintézést igénylő műveleteket különböző szoftvereken keresztül végezzék el.

Mivel egy nagyobb szervezetben a munkavállalók száma akár a több százat is elérheti, és ezekre a személyekre meglehet, hogy külön szabályok vonatkoznak így szükségessé vált a szabadság nyilvántartó rendszerek bevezetése.
Ezt az igényt természetesen hamar sikerült kielégíteni, de a piacon található szolgáltatások általános megoldást kínálnak amik a lehető legtöbb szervezeti egységre alkalmazhatóak és a legelterjedtebb szabályozásokat fedik le csupán.

Ezeket az igényeket eleinte jól megtervezett excel táblák és sablonok elégítették ki, de idővel ezek bonyolulttá és nehezen követhetővé váltak, valamint kimondottan sok helyet foglaltak. nem kellett sok idő, hogy a WEB applikációk megjelenésével és széles körben elterjedésével, több piac orientált rendszer szülessen.

A létező megoldások általánossága miatt, nem lehet specifikusan egy adott ország törvényeire alkalmazva konfigurálni, így a megannyi eltérő szabályozás miatt minden évben az alkalmazott számára elérhető szabadnapok kiszámolása, és jóváírása többnyire kézzel történik. Továbbá felmerül az a probléma is, hogy amennyiben egy alkalmazottra, vagy szervezeti egységre nem ugyanazok a szabályok vonatkoznak, abban az esetben milyen beállítások érhetőek el a különbségek kezelésére? Ezt a kérdést megválaszolva azt vesszük észre, hogy jelenleg nem elérhető olyan WEB-es szoftver ami alkalmazható vagy akár, automatizálható egy Magyar felsőoktatási intézményben, anélkül, hogy az eddigi papírmunkát maradéktalanul leváltsa. Ezáltal a felhasználók kénytelenek nehezen átlátható módon igényelni és a követni az elérhető és felhasznált szabadságaikat.

Az egyetemeken központosított nyilvántartást vezetnek az alkalmazottak szabadságairól, de ezzel kapcsolatban a következő kérdések merülnek fel: Vajon minden alkalmazottra egyforma jogszabályok érvényesek? A különböző jogszabályokat, képes külön kezelni a rendszer? Mennyire hozzáférhető ez az alkalmazottak számára? Nyomon tudják követni egyszerűen a felhasznált és hátralévő szabadnapjaikat az emberek? Mennyi időt vesz el tőlük egy ilyen ügyintézés? Van olyan lehetőség ami mindenkinek megfelel?

Ezen kérdéskört kutatva a dolgozatom célja, hogy tervezzek és megvalósítsak egy olyan WEB-es felületen elérhető alkalmazást, ami egy központi szerverre telepítve, majd a megfelelő beállítások után, lehetővé teszi, hogy a Magyar szabályozásoknak megfelelően a felhasználók képesek legyenek gond nélkül naprakész információhoz jutni a szabadságaikkal kapcsolatban. Valamint a rendelkezésre álló adatok alapján a lehető legnagyobb mértékben automatizáljam a rendszert. 

Dolgozatom során részletesen utánajártam az országunkban létező szabadság típusoknak, ezekhez kapcsolódó feltételeknek, valamint az átlagostól eltérő munkakörökre vonatkozó szabályozásoknak és implementáltam egy általam tervezett megoldást. Az implementáció során a próbáltam céges körülményeket szimulálni és a lehető legmodernebb technikai megoldásokkal dolgozni. Szakdolgozatom a bonyolult struktúra, az eltérő vagy akár változó statisztikai lehetőségek és az igényes megvalósítás időigényessége miatt nem tér ki a felsőoktatásban dolgozó közalkalmazottakra nem releváns pótszabadságokra valamint a szülési szabadság kimondottan bonyolult lehetséges kimeneteleire, csak és kizárólag az ideális esetre, továbbá a megvalósítás esetén: statisztikai funkciókra, a felettesi jóváhagyásra, valamint a felületre. Az felsorolt működési elemekre csak elméleti, rendszertervezés szintjén tér ki.

A kutatás és az implementáció remélhetőleg választ ad arra, hogy lehetséges-e megvalósítani teljesen automatizált szabadságnyilvántartó applikációkat, valamint az általam megírt szoftver milyen további fejlesztésekkel, alkalmazható akár napi szinten a Magyar egyetemek tanszékein dolgozó oktatók, adminisztrátorok és egyéb munkakörökben dolgozók átláthatóbb szabadság menedzsmentjének érdekében.  