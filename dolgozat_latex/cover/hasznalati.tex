\pagestyle{empty}

\noindent \textbf{\Large Adathordozó használati útmutató}

\vskip 1cm

Ez jellemzően csak egy fél-egy oldalas leírás.
Arra szolgál, hogy ha valaki kézhez kapja a szakdolgozathoz tartozó CD-t, akkor tudja, hogy mi hol van rajta.
Jellemzően elég csak felsorolni, hogy milyen jegyzékek vannak, és azokban mi található.
Az elkészített programok telepítéséhez, futtatásához tartozó instrukciók kerülhetnek ide.

A CD lemez tartalma
\begin{itemize}
\item a \texttt{dolgozat.pdf} fájl,
\item a \textbf{dolgozat\_latex} mappa tartalmazza a LaTeX forráskódját a dolgozatnak,
\item a \textbf{program} mappa tartalmazza az elkészített program forráskódját
\end{itemize}

\vskip 1cm

A program futtatásához a következő előfeltételek szükségesek:
\begin{itemize}
	\item A futtató számítógépre telepített Docker környezet
	\item Legalább Java 11-es verzió.
	\item Configurált Maven
	\item Node.js
	\item Angular.js
\end{itemize}

Az előfeltétel teljesítése után a következő parancsokkal indítható a program.

\begin{itemize}
	\item a holiday\_registry könyvtárban állva console-ban kiadjuk az \textbf{mvn clean install} parancsot
	\item a Frontend demo indításához a frontend mappában consoleon keresztül kiadjuk az \textbf{ng serve} parancsot (Ezután a localhost:4000 portján elérhető az oldal)
	\item a docker könyvtárban consoleból kiadjuk a \textbf{docker-compose up} parancsot.
	\item a Backend applikáció futásához a web könyvtárból kiandjuk az \textbf{mvn spring-boot:run} parancsot a program felülete a http://localhost:8080/swagger-ui/ linken elérhető.
\end{itemize}