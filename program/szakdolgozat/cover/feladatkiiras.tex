%Feladatkiiras
\begin{flushleft}
\textsc{\bfseries Miskolci Egyetem}\\
Gépészmérnöki és Informatikai Kar\\
Alkalmazott Matematikai Intézeti Tanszék\hspace*{4cm}\hfil \textbf{Szám:}
\end{flushleft}
\vskip 0.5cm
\begin{center}
\large\textsc{\bfseries Szakdolgozat Feladat}
\end{center}
\vskip 0.5cm
Strutinszky Alex Roland (PXMQ3P) programtervező informatikus jelölt részére.\newline

\noindent\textbf{A szakdolgozat tárgyköre:} WEB-es alkalmazás, automatizálás, szabadság, nyilvántartás\newline

\noindent\textbf{A szakdolgozat címe:} Automatizált szabadságnyilvántartó rendszer tervezése és megvalósítása WEB-es környezetben.\newline

\noindent\textbf{A feladat részletezése:}

\medskip

\emph{
	\begin{enumerate}
		\item {Ismerje meg a Miskolci Egyetem Szabályzatának a munkahelyi szabadság kiadására vonatkozó pontjait!}
		\item {Készítse el a rendszertervét egy automatikus havi lejelentések megtételére képes éves szabadságnyilvántartó rendszernek! A program vegye figyelembe a rendes és eseti szabadnapokat, tegye lehetővé a közalkalmazott szabadságválasztásában a maximális szabadságfokot, ugyanakkor tartsa be az összes vonatkozó országos és helyi szabályzat előírását. A rendszer webes felületen, kliens-szerver rendszerben működjön, kezelje az alkalmazotti, vezetői és rendszergazdai jogosultságokat. Nyújtson áttekintő felületet az alkalmazott saját szabadságainak állására, továbbá a vezető számára az összes alkalmazott vonatkozásában.}
		\item {Válassza meg az alkalmas programozási és futtatókörnyezetet a program elkészítéséhez, választását indokolja!}
		\item {Készítse el az alkalmazás programját és minden eshetőségre kiterjedő képzeletbeli adatokkal tesztelje!}
		\item {Készítse el az alkalmazás felhasználói kézikönyvét és ennek kivonatos ismertetésével mutassa be a használat lépéseit!}
	\end{enumerate}
	}

\medskip

\vfill

\noindent\textbf{Témavezető:} Dr. Dudás László PhD (a műszaki tudomány kandidátusa, egyetemi docens) \newline

% \noindent\textbf{Konzulens(ek):} (akkor kötelezõ, ha a témavezetõ nem valamelyik matematikai tanszékrõl való; de persze lehet egyébként is)\newline

\noindent\textbf{A feladat kiadásának ideje: 2018. 09. 01.}\newline

%\noindent\textbf{A feladat beadásának határideje:}

\vskip 2cm

\hbox to \hsize{\hfil{\hbox to 6cm {\dotfill}\hbox to 1cm{}}}

\hbox to \hsize{\hfil\hbox to 3cm {szakfelelős}\hbox to 2cm{}}

\newpage

\vspace*{1cm}  
\begin{center}
\large\textsc{\bfseries Eredetiségi Nyilatkozat}
\end{center}
\vspace*{2cm}  

Alulírott \textbf{Strutinszky Alex Roland}; Neptun-kód: \texttt{PXMQ3P} a Miskolci Egyetem Gépészmérnöki és Informatikai Karának végzős Programtervező informatikus szakos hallgatója ezennel büntetőjogi és fegyelmi felelősségem tudatában nyilatkozom és aláírásommal igazolom, hogy \textit{Automatizált szabadságnyilvántartó rendszer tervezése és megvalósítása WEB-es környezetben.}
című szakdolgozatom saját, önálló munkám; az abban hivatkozott szakirodalom
felhasználása a forráskezelés szabályai szerint történt.\\

Tudomásul veszem, hogy szakdolgozat esetén plágiumnak számít:
\begin{itemize}
\item szószerinti idézet közlése idézőjel és hivatkozás megjelölése nélkül;
\item tartalmi idézet hivatkozás megjelölése nélkül;
\item más publikált gondolatainak saját gondolatként való feltüntetése.
\end{itemize}

Alulírott kijelentem, hogy a plágium fogalmát megismertem, és tudomásul veszem, hogy
plágium esetén szakdolgozatom visszautasításra kerül.

\vspace*{3cm}

\noindent Miskolc, \hbox to 2cm{\dotfill} .év \hbox to 2cm{\dotfill} .hó \hbox to 2cm{\dotfill} .nap

\vspace*{3cm}

\hspace*{8cm}\begin{tabular}{c}
\hbox to 6cm{\dotfill}\\
Hallgató
\end{tabular}



\newpage

\noindent 1.

\begin{tabular}{cl}
&szükséges (módosítás külön lapon) \\
A szakdolgozat feladat módosítása& \\
& nem szükséges\\
&\\
\hbox to 4cm{\dotfill}&\multicolumn{1}{c}{\hbox to 5cm{\dotfill}}\\
dátum& \multicolumn{1}{c}{témavezető(k)}
\end{tabular}
\vskip1.5mm

\noindent 2. A feladat kidolgozását ellenőriztem:

\vskip1.5mm

\begin{tabular}{l@{\hspace*{4cm}}l}
témavezető (dátum, aláírás):& konzulens (dátum, aláírás):\\
\dotfill&\dotfill\\
\dotfill&\dotfill\\
\dotfill&\dotfill
\end{tabular}

\vskip1.5mm

\noindent 3. A szakdolgozat beadható:

\vskip1.5mm

\begin{tabular}{@{\hspace*{1.3cm}}c@{\hspace*{2.1cm}}c}
\hbox to 4cm{\dotfill}&\multicolumn{1}{c}{\hbox to 5cm{\dotfill}}\\
dátum& \multicolumn{1}{c}{témavezető(k)}
\end{tabular}

\vskip1.5mm

\noindent 4.
\begin{tabular}[t]{@{}l@{\hspace*{1mm}}l@{\hspace*{1mm}}l@{}}
A szakdolgozat& \hbox to 3.5cm{\dotfill} &szövegoldalt\\
              & \hbox to 3.5cm{\dotfill} &program protokollt (listát, felhasználói leírást)\\
              &\hbox to 3.5cm{\dotfill}   &elektronikus adathordozót (részletezve)\\
              &\hbox to 3.5cm{\dotfill} & \\
              &\hbox to 3.5cm{\dotfill} &egyéb mellékletet (részletezve)\\
              &\hbox to 3.5cm{\dotfill} &\\
\end{tabular}
\newline tartalmaz.

\vskip1.5mm

\begin{tabular}{@{\hspace*{1.3cm}}c@{\hspace*{2.1cm}}c}
\hbox to 4cm{\dotfill}&\multicolumn{1}{c}{\hbox to 5cm{\dotfill}}\\
dátum& \multicolumn{1}{c}{témavezető(k)}
\end{tabular}

\noindent 5.

\begin{tabular}{ll}
&bocsátható\\
A szakdolgozat bírálatra& \\
& nem bocsátható\\
\end{tabular}

\vskip1.5mm

\noindent A bíráló neve: \hbox to 8cm{\dotfill}

\vskip4mm

\begin{tabular}{@{\hspace*{1.3cm}}c@{\hspace*{2.1cm}}c}
\hbox to 4cm{\dotfill}&\multicolumn{1}{c}{\hbox to 5cm{\dotfill}}\\
dátum& \multicolumn{1}{c}{szakfelelős}
\end{tabular}

\noindent 6.
\begin{tabular}[t]{@{}l@{\hspace*{1mm}}l@{\hspace*{1mm}}l@{}}
A szakdolgozat osztályzata& &\\
&a témavezető javaslata:& \hbox to 3cm{\dotfill}\\
&a bíráló javaslata:& \hbox to 3cm{\dotfill}\\
&a szakdolgozat végleges eredménye:& \hbox to 3cm{\dotfill}
\end{tabular}

\vspace*{4mm}

\noindent Miskolc, \hbox to 4.5cm{\dotfill} \hspace*{2.5cm}
\begin{tabular}[t]{cc}
\hbox to 6cm{\dotfill}\\
a Záróvizsga Bizottság Elnöke
\end{tabular}
